\documentclass[12pt]{article}
\usepackage[utf8]{inputenc}
\usepackage[letterpaper,margin=0.75in]{geometry}
\renewcommand{\rmdefault}{ptm}
\usepackage[numbers]{natbib}
\usepackage{hyperref}

\title{Classification multiclasses: légumes secs}
\author{
  Team : Kazou\\
  Virgile Retault : 2164296\\
  Sebastien Foucher : 2162248
}

\vspace{-5ex}

\begin{document}

\maketitle

\section*{Méthodologie}
\newpage

\begin{itemize}
	\item lorem ipsum
	
\end{itemize}


\section*{Résultats et Evolution de l'agent}

Voici les scores obtenus avec ces différentes méthodes:
\begin{center}
	\begin{tabular}{ |c|c| }
		\hline
		Méthode & Précision \\\hline\hline
	  K-Neighbors Classifier & 80\% \\\hline
		Decision Tree & 89.6\% \\\hline
		SVM avec noyau linéaire & 90.1\% \\\hline
		Gradient Boosting (avec PCA 12) & 91.3\% \\\hline
		Random Forest (sans PCA) & 91.5\% \\\hline
		Réseau de neurone (12, 6, 6, 7) (avec PCA 12) & 92.2\% \\\hline
		Adaboost avec random forest (avec PCA 12) & 92.9\% \\\hline
		StackClassifier combinant RandomForest, ExtraTree, AdaBoost (avec PCA 12) & 93.3\% \\\hline
		Random Forest (avec PCA 12) & 93.4\% \\\hline

	\end{tabular} 
\end{center}


\section*{Discussion}

Lorem ipsum

\end{document}

